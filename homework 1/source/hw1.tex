\documentclass{article}
\usepackage[utf8]{inputenc}

\title{Computational Physics (physics760) \\ Exercise 1}
\author{Ajay S. Sakthivasan, Dongjin Suh}
\date{October 28, 2022}

\begin{document}

\maketitle

\section{Simulation of the 1D Ising Model}
\begin{enumerate}
    \item $J$ is the interaction coefficient, and determines the strength of interaction between two adjacent lattice points. It's apparent from the Hamiltonian that $J = 0$ corresponds to a system in which there's no interaction between different points in the lattice. In such cases, the Hamiltonian has only one possible non-zero contribution, which is from the energy due to the external field. Further, $J>0$ corresponds to ferromagnets, where the spins desire to be aligned (neighbouring spins have same signs). And, $J<0$ corresponds to antiferromagnets, where the spins desire to anti-aligned (neighbouring spins have opposite signs).
\end{enumerate}

\end{document}