\documentclass{article}
\usepackage[utf8]{inputenc}
\usepackage{graphicx}
\usepackage{amsmath}
\usepackage{amssymb}
\usepackage{amsthm}
\usepackage{bm}

\title{Computational Physics (physics760)\\Exercise 6}
\author{Ajay S. Sakthivasan, Dongjin Suh}
\date{December 9, 2022}

\begin{document}

\maketitle

\begin{enumerate}

\item   \textbf{Correlator Bias}\\
As discussed in the lecture, the algorithm may not explore all spin configurations for large ($\gg J_c$) value of $J$. The algorithm might settle for one particular spin configuration, either almost all spin-up or almost all spin-down, in which case the correlator will always be close to $1$. This means that there is no bias in the Correlator, since we expect for large values of $J$, the correlation to be close to $1$.

\item \textbf{Correlator ($r = 0$)}\\
If $r=0$, we are calculating the correlation o each spin site with itself. Since, each spin can either up or down, the product will always be $1$. Dividing by the total number of sites, we get $C_0 = 1$.



\end{enumerate}
\end{document}
